\documentclass[10pt,twoside]{article}\usepackage[]{graphicx}\usepackage[dvipsnames,svgnames,table]{xcolor}
% maxwidth is the original width if it is less than linewidth
% otherwise use linewidth (to make sure the graphics do not exceed the margin)
\makeatletter
\def\maxwidth{ %
  \ifdim\Gin@nat@width>\linewidth
    \linewidth
  \else
    \Gin@nat@width
  \fi
}
\makeatother

\definecolor{fgcolor}{rgb}{0.345, 0.345, 0.345}
\newcommand{\hlnum}[1]{\textcolor[rgb]{0.686,0.059,0.569}{#1}}%
\newcommand{\hlsng}[1]{\textcolor[rgb]{0.192,0.494,0.8}{#1}}%
\newcommand{\hlcom}[1]{\textcolor[rgb]{0.678,0.584,0.686}{\textit{#1}}}%
\newcommand{\hlopt}[1]{\textcolor[rgb]{0,0,0}{#1}}%
\newcommand{\hldef}[1]{\textcolor[rgb]{0.345,0.345,0.345}{#1}}%
\newcommand{\hlkwa}[1]{\textcolor[rgb]{0.161,0.373,0.58}{\textbf{#1}}}%
\newcommand{\hlkwb}[1]{\textcolor[rgb]{0.69,0.353,0.396}{#1}}%
\newcommand{\hlkwc}[1]{\textcolor[rgb]{0.333,0.667,0.333}{#1}}%
\newcommand{\hlkwd}[1]{\textcolor[rgb]{0.737,0.353,0.396}{\textbf{#1}}}%
\let\hlipl\hlkwb

\usepackage{framed}
\makeatletter
\newenvironment{kframe}{%
 \def\at@end@of@kframe{}%
 \ifinner\ifhmode%
  \def\at@end@of@kframe{\end{minipage}}%
  \begin{minipage}{\columnwidth}%
 \fi\fi%
 \def\FrameCommand##1{\hskip\@totalleftmargin \hskip-\fboxsep
 \colorbox{shadecolor}{##1}\hskip-\fboxsep
     % There is no \\@totalrightmargin, so:
     \hskip-\linewidth \hskip-\@totalleftmargin \hskip\columnwidth}%
 \MakeFramed {\advance\hsize-\width
   \@totalleftmargin\z@ \linewidth\hsize
   \@setminipage}}%
 {\par\unskip\endMakeFramed%
 \at@end@of@kframe}
\makeatother

\definecolor{shadecolor}{rgb}{.97, .97, .97}
\definecolor{messagecolor}{rgb}{0, 0, 0}
\definecolor{warningcolor}{rgb}{1, 0, 1}
\definecolor{errorcolor}{rgb}{1, 0, 0}
\newenvironment{knitrout}{}{} % an empty environment to be redefined in TeX

\usepackage{alltt}
\usepackage[marginparsep=1em]{geometry}
\geometry{lmargin=1.0in,rmargin=1.0in, bmargin=1.2in,  tmargin=1.2in}
\usepackage[dvipsnames,svgnames,table]{xcolor}
\usepackage{graphicx}
\usepackage{amssymb}
\usepackage{epstopdf}
\usepackage{verbatim}
\usepackage{enumerate}
\usepackage{bm}
\usepackage{amsthm}
\usepackage{float}
\usepackage{amsmath}
\usepackage{fancyhdr}
\usepackage{hyperref}
\usepackage{mathtools}
            
%%%%  SHORTCUT COMMANDS  %%%%
\newcommand{\ds}{\displaystyle}
\newcommand{\Z}{\mathbb{Z}}
\newcommand{\T}{\mathcal{T}}
\newcommand{\arc}{\rightarrow}
\newcommand{\R}{\mathbb{R}}
\newcommand{\RP}{\mathbb{R}(+)}
\newcommand{\Rs}{\mathbb{R}^{**}}
\newcommand{\C}{\mathbb{C}}
\newcommand{\E}{\mathbb{E}}
\newcommand{\B}{\mathcal{B}}
\newcommand{\PX}{\mathcal{P}(X)}
\newcommand*\rot{\rotatebox{90}}
\newcommand*\OK{\ding{51}}
\newcommand{\N}{\mathbb{N}}
\newcommand{\Q}{\mathbb{Q}}
\newcommand{\Answer}{\vspace{2mm}\textbf{\underline{Answer}}\\}
\newcolumntype{L}{>{$}l<{$}}
\newcolumntype{C}{>{$}c<{$}}
\newcommand{\GL}{\text{GL}}
\newcommand{\SL}{\text{SL}}

\newcommand{\stirling}[2]{\genfrac{\{}{\}}{0pt}{}{#1}{#2}}

\pagestyle{fancy}
\fancyhf{}
\renewcommand{\sectionmark}[1]{\markright{\thesection.\ #1}}
\lhead{\fancyplain{}{}} 
\fancyhead[RE,RO]{Math 4451, Spring 2026}
\fancyfoot[RE,RO]{\thepage}
\IfFileExists{upquote.sty}{\usepackage{upquote}}{}
\begin{document}
\begin{flushright}
\begin{minipage}{.33\textwidth}
\rightline{YOUR NAME}
\rightline{\href{mailto:YOUR EMAIL}{YOUR EMAIL}}
\rightline{\today}
\end{minipage}
\end{flushright}

\begin{center}
{\large{\textbf{Homework 2}}}
\end{center}

\begin{enumerate}
    \item (from last HW, 3 points) Suppose that $X$ is a discrete random variable with:
    $$
    P(X = x) = \begin{cases} \frac{2}{3}\theta & x = 0 \\ \frac{1}{3}\theta & x = 1 \\ \frac{2}{3}(1 - \theta) & x = 2 \\ \frac{1}{3}(1 - \theta) & x = 3 \\\end{cases}
    $$
    where $0 \leq \theta \leq 1$. Suppose we observe 10 independent observations from such a distribution: $(3, 0, 2, 1, 3, 2, 1, 0, 2, 1)$.
    (We'll return to this same distribution later)
    \begin{itemize}
      \item (2 points) Find the MLE of $\theta$.
      \item (1 point) Find an approximate standard error for your estimate.
    \end{itemize}
    \item Often, real data is messy, and may have missing information. In the following examples, we consider finding the MLE in these types of scenarios.
    \begin{enumerate}
      \item (2 points) We believe a particular measurement is approximately normally distributed. However, we are only able to measure accuracy up to integer values. Thus, we observe 3 partial values: $1 \leq x^*_1 < 2$, $3 \leq x_2^{*} < 4$, and $6 \leq x_3^* < 7$, only knowing the interval into which the observations fall rather than the specific value. Find an expression of the likelihood function under this model (Hint: You can use the $\Phi(x)$ notation for the CDF of a standard normal.)
      \item (3 points) Suppose 100 seeds are planted, and it is known only that $x^*\leq 10$ seeds germinated---the exact number of germinating seeds is unknown. Let $\theta$ be the probability that seed $i$ germinates (the $i$th seed will either germinate, or it will not), and we assume that seeds are independent. What is the likelihood function of this model? Find a maximum likelihood estimate. Plot the likelihood function, and comment about the point estimate (the MLE) and the likelihood function.
    \end{enumerate}
    
    \item (3 points) \textbf{Ecology: capture / recapture}. A common approach to estimate the number of animals $N$ in a given population is the following: Capture a subset of the population, mark them with tags and release them into the wild. After some time, we recapture a sample from the population of size $N$, and calculate the proportion of re-captured animals that are marked (previously captured), and use this to estimate the total population size. 
    
    Suppose we want to estimate the number of badgers $N$ in a region. We capture and tag $N_1 = 25$ animals, and release them. Later, we capture $n = 60$ badgers, and we find $n_1 = 5$ tagged animals, and $n_2 = 55$ non-tagged animals. Assuming the badgers were caught randomly, use a hypergeometric distribution to get find the likelihood, as a function of $\theta = N$. Find an MLE, and plot plot the likelihood function. 
\end{enumerate}


\end{document}
