\input{../header}

% \mode<beamer>{\usetheme{AnnArbor}}
\mode<beamer>{\usetheme{metropolis}}
\mode<beamer>{\metroset{block=fill}}
% \mode<beamer>{\usecolortheme{wolverine}}

\mode<beamer>{\setbeamertemplate{section in toc}[sections numbered]}
\mode<beamer>{\setbeamertemplate{subsection in toc}[subsections numbered indented]}

% \mode<beamer>{\usefonttheme{serif}}
\mode<beamer>{\setbeamertemplate{footline}}
\mode<beamer>{\setbeamertemplate{footline}[frame number]}
\mode<beamer>{\setbeamertemplate{frametitle continuation}[from second][\insertcontinuationcountroman]}
\mode<beamer>{\setbeamertemplate{navigation symbols}{}}

\mode<handout>{\pgfpagesuselayout{2 on 1}[letterpaper,border shrink=5mm]}

\newcommand\CHAPTER{1}
% \newcommand\answer[2]{\textcolor{blue}{#2}} % to show answers
% \newcommand\answer[2]{\textcolor{red}{#2}} % to show answers
 \newcommand\answer[2]{#1} % to show blank space

\title{\vspace{2mm} \link{https://jeswheel.github.io/4451_f25/}{Mathematical Statistics II}\\ \vspace{2mm}
Chapter \CHAPTER: Probability}
\author{Jesse Wheeler}
\date{}

\setbeamertemplate{footline}[frame number]




\begin{document}

\maketitle

\mode<article>{\tableofcontents}

\mode<presentation>{
  \begin{frame}{Outline}
    \tableofcontents
  \end{frame}
}

\begin{frame}{Course Overview}
  \begin{itemize}
  \item This course is the first part of a two semester introductory course on Mathematical Statistics.
  \item Our goal is to cover Chapters 1-10 of ``Mathematical Statistics and Data Analysis", by John A. Rice \citep{rice07}.
  \item Topics include: Probability, Random Variables, Discrete and Continuous distributions, Order Statistics, Limit Theorems, Point and Interval Estimation, Uniformly most powerful tests, likelihood ratio tests, chi-square and F tests, and nonparameteric tests.
  \item Roughly speaking, 4450 and 4451 can be broken into two parts: 
  \begin{itemize}
    \item Math 4450: Probability (mathematics of randomness)
    \item Math 4451: Statistics (procedures for analyzing data)
  \end{itemize}
  \end{itemize}
\end{frame}

\subsection{Logistics}

\begin{frame}{Course Logistics}
  \begin{itemize}
    \item \link{https://jeswheel.github.io/4451\_s26/Introduction.pdf}{About Me}
    \item Course Website: \link{https://jeswheel.github.io/4451\_s26/}{https://jeswheel.github.io/4451\_f25/}. 
    \item Canvas: Canvas will be used to submit assignments, view grades, and for course announcements.
    \item \link{https://jeswheel.github.io/4451\_s26/syllabus.pdf}{Course Syllabus}
    \item \link{https://jeswheel.github.io/4451\_s26/rubric\_homework.html}{Homework grading rubric}.
  \end{itemize}
\end{frame}


\newcommand\acknowledgments{
\begin{itemize}
\item   Compiled on {\today} using \Rlanguage version 4.5.2.
\item   \parbox[t]{0.75\textwidth}{Licensed under the \link{http://creativecommons.org/licenses/by-nc/4.0/}{Creative Commons Attribution-NonCommercial license}.
    Please share and remix non-commercially, mentioning its origin.}
    \parbox[c]{1.5cm}{\includegraphics[height=12pt]{../cc-by-nc}}
\item We acknowledge \link{https://jeswheel.github.io/4451_s26/acknowledge.html}{students and instructors for previous versions of this course / slides}.
\end{itemize}
}

\mode<presentation>{
\begin{frame}[allowframebreaks=0.8]{References and Acknowledgements}
  
\bibliography{../bib4451}

\vspace{3mm}

\acknowledgments

\end{frame}
}

\mode<article>{

\newpage

{\bf \Large \noindent Acknowledgments}

\acknowledgments

\newpage

\bibliography{../bib4451}

}



\end{document}







