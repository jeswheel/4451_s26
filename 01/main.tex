\input{../header}

% \mode<beamer>{\usetheme{AnnArbor}}
\mode<beamer>{\usetheme{metropolis}}
\mode<beamer>{\metroset{block=fill}}
% \mode<beamer>{\usecolortheme{wolverine}}

\mode<beamer>{\setbeamertemplate{section in toc}[sections numbered]}
\mode<beamer>{\setbeamertemplate{subsection in toc}[subsections numbered indented]}

% \mode<beamer>{\usefonttheme{serif}}
\mode<beamer>{\setbeamertemplate{footline}}
\mode<beamer>{\setbeamertemplate{footline}[frame number]}
\mode<beamer>{\setbeamertemplate{frametitle continuation}[from second][\insertcontinuationcountroman]}
\mode<beamer>{\setbeamertemplate{navigation symbols}{}}

\mode<handout>{\pgfpagesuselayout{2 on 1}[letterpaper,border shrink=5mm]}

\newcommand\CHAPTER{1}
% \newcommand\answer[2]{\textcolor{blue}{#2}} % to show answers
% \newcommand\answer[2]{\textcolor{red}{#2}} % to show answers
 \newcommand\answer[2]{#1} % to show blank space

\title{\vspace{2mm} \link{https://jeswheel.github.io/4451_f25/}{Mathematical Statistics II}\\ \vspace{2mm}
Chapter \CHAPTER: Probability}
\author{Jesse Wheeler}
\date{}

\setbeamertemplate{footline}[frame number]




\begin{document}

\maketitle

\mode<article>{\tableofcontents}

% \mode<presentation>{
%   \begin{frame}{Outline}
%     \tableofcontents
%   \end{frame}
% }

\begin{frame}[allowframebreaks]{Course Overview}
  \begin{itemize}
    \item The larger focus of last semester (Math 4450) was probability.
    \item Though we continue where we left off, this semester (Math 4451) will have a much stronger focus on statistics.
    \item Both probability and statistics are, fundamentally, the study of with randomness... what's the difference? \mode<article>{\textbf{Depends on who you ask!}}
    
    \framebreak
    
    \mode<presentation>{
      \vspace{2cm}
    }
    
    \begin{quote}
      ``The science of collecting, displaying, and analysing data." -- \citet{OxfordStat}
    \end{quote}
    
    \mode<presentation>{
      \vfill
    }
    
    \begin{quote}
      ``The discipline that concerns the collection, organization, analysis, interpretation, and presentation of data." -- \citet{statsWiki}
    \end{quote}
    
    \mode<presentation>{
      \vfill
    }
    
    \begin{quote}
    Something like: ``The study of extracting useful information from data in a rigorous way." -- Me (it's hard to define an entire discipline).
    \end{quote}
    
  \end{itemize}
\end{frame}

\begin{frame}[allowframebreaks]{Probability vs Statistics}

\begin{itemize}
  \item Any of the above definitions (accurately) suggests that probability is a key part of statistics. So where do we draw the line? Does it matter?
  \item \citet{pawitan01} dichotomizes the difference in terms of \emph{deductive} vs \emph{inductive} reasoning.
  \item Roughly speaking, \emph{deductive} arguments moves from general principles (assumptions) to make specific conclusions.
  In \emph{inductive} reasoning, we use specific observations (data) to make broader generalizations.
\end{itemize}

\framebreak

\begin{exampleblock}{Traffic Accidents}
  Suppose we are interested in the random quantity $X_i$, the number of accidents during week $i$ at a particular intersection.
  From last semester, a common model for this situation is a Poisson-process.
\end{exampleblock}

\begin{itemize}
  \item \emph{Probability (deductive)}: If $X_i$ follows a Poisson$(\lambda)$ distribution (general principle), then what is the expected number of accidents per week (specific conclusion)? What is the probability that we observe more than $10$ accidents?
  
  \item \emph{Statistics (inductive)}: Suppose we count the number of accidents over a 6 week period, observing: $3, 4, 2, 7, 3, 3$ accidents (specific observations).
  What value $\lambda$ might describe the Poisson-process that generated the data (broader generalization)? Is the Poisson assumption reasonable given the data?
\end{itemize}

\begin{itemize}
  \item From the example above, we can see both ideas used in conjunction for making informed decisions.
  \item Many statistics problems rely on deductive reasoning in probability, geometry, topology, analysis, etc. to build theory for ways of performing inductive reasoning with specific observations (data).
  \item Another example related to my own research in population modeling...
\end{itemize}



\begin{figure}[ht]
\includegraphics[width=\textwidth]{ebola.png}
\end{figure}

\begin{itemize}
  \item {\small (Statistics) Given the data (specific example), what can we learn about the dynamic system / generative process (generalization)?}
  \item {\small (Probability) Under our assumed process (assumed principle), what is our prediction for the Ebola burden over the next year (specific conclusion)?}
\end{itemize}

\end{frame}

\begin{frame}[allowframebreaks]{Statistics and Math 4451}

  \begin{itemize}
  \item \citet{pawitan01} further categorizes statistics in terms of five key `statistical activities' in the preface of his book:
    \begin{itemize}
      \item \emph{Planning}: making decisions about the study design or sampling protocol, what measurements to take, stratification, sample size, etc.
      \item \emph{Describing}: summarizing the bulk of data in few quantities, finding or revealing meaningful patterns or trends, etc. 
      \item \emph{Modeling}: developing mathematical models with few parameters to represent the patterns, or to explain the variability in terms of relationship between variables.
      \item \emph{Inference}: assessing whether we are seeing a real or spurious pattern or relationship, which typically involves an evaluation of the uncertainty in the parameter estimates.
      \item \emph{Model Checking}: assessing whether the model is sensible for the data.
    \end{itemize}
  \item A lot of early statistics were focused on the first two activities: \emph{planning} and \emph{describing}.
  We will not spend much time this semester discussing methods related to these two activities.
  \end{itemize}
\end{frame}

\newcommand\acknowledgments{
\begin{itemize}
\item   Compiled on {\today} using \Rlanguage version 4.5.2.
\item   \parbox[t]{0.75\textwidth}{Licensed under the \link{http://creativecommons.org/licenses/by-nc/4.0/}{Creative Commons Attribution-NonCommercial license}.
    Please share and remix non-commercially, mentioning its origin.}
    \parbox[c]{1.5cm}{\includegraphics[height=12pt]{../cc-by-nc}}
\item We acknowledge \link{https://jeswheel.github.io/4451_s26/acknowledge.html}{students and instructors for previous versions of this course / slides}.
\end{itemize}
}

\mode<presentation>{
\begin{frame}[allowframebreaks=0.8]{References and Acknowledgements}
  
\bibliography{../bib4451}

\vspace{3mm}

\acknowledgments

\end{frame}
}

\mode<article>{

\newpage

{\bf \Large \noindent Acknowledgments}

\acknowledgments

\newpage

\bibliography{../bib4451}

}



\end{document}







