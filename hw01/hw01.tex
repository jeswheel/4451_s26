\documentclass[10pt,twoside]{article}\usepackage[]{graphicx}\usepackage[dvipsnames,svgnames,table]{xcolor}
% maxwidth is the original width if it is less than linewidth
% otherwise use linewidth (to make sure the graphics do not exceed the margin)
\makeatletter
\def\maxwidth{ %
  \ifdim\Gin@nat@width>\linewidth
    \linewidth
  \else
    \Gin@nat@width
  \fi
}
\makeatother

\definecolor{fgcolor}{rgb}{0.345, 0.345, 0.345}
\newcommand{\hlnum}[1]{\textcolor[rgb]{0.686,0.059,0.569}{#1}}%
\newcommand{\hlsng}[1]{\textcolor[rgb]{0.192,0.494,0.8}{#1}}%
\newcommand{\hlcom}[1]{\textcolor[rgb]{0.678,0.584,0.686}{\textit{#1}}}%
\newcommand{\hlopt}[1]{\textcolor[rgb]{0,0,0}{#1}}%
\newcommand{\hldef}[1]{\textcolor[rgb]{0.345,0.345,0.345}{#1}}%
\newcommand{\hlkwa}[1]{\textcolor[rgb]{0.161,0.373,0.58}{\textbf{#1}}}%
\newcommand{\hlkwb}[1]{\textcolor[rgb]{0.69,0.353,0.396}{#1}}%
\newcommand{\hlkwc}[1]{\textcolor[rgb]{0.333,0.667,0.333}{#1}}%
\newcommand{\hlkwd}[1]{\textcolor[rgb]{0.737,0.353,0.396}{\textbf{#1}}}%
\let\hlipl\hlkwb

\usepackage{framed}
\makeatletter
\newenvironment{kframe}{%
 \def\at@end@of@kframe{}%
 \ifinner\ifhmode%
  \def\at@end@of@kframe{\end{minipage}}%
  \begin{minipage}{\columnwidth}%
 \fi\fi%
 \def\FrameCommand##1{\hskip\@totalleftmargin \hskip-\fboxsep
 \colorbox{shadecolor}{##1}\hskip-\fboxsep
     % There is no \\@totalrightmargin, so:
     \hskip-\linewidth \hskip-\@totalleftmargin \hskip\columnwidth}%
 \MakeFramed {\advance\hsize-\width
   \@totalleftmargin\z@ \linewidth\hsize
   \@setminipage}}%
 {\par\unskip\endMakeFramed%
 \at@end@of@kframe}
\makeatother

\definecolor{shadecolor}{rgb}{.97, .97, .97}
\definecolor{messagecolor}{rgb}{0, 0, 0}
\definecolor{warningcolor}{rgb}{1, 0, 1}
\definecolor{errorcolor}{rgb}{1, 0, 0}
\newenvironment{knitrout}{}{} % an empty environment to be redefined in TeX

\usepackage{alltt}
\usepackage[marginparsep=1em]{geometry}
\geometry{lmargin=1.0in,rmargin=1.0in, bmargin=1.2in,  tmargin=1.2in}
\usepackage[dvipsnames,svgnames,table]{xcolor}
\usepackage{graphicx}
\usepackage{amssymb}
\usepackage{epstopdf}
\usepackage{verbatim}
\usepackage{enumerate}
\usepackage{bm}
\usepackage{amsthm}
\usepackage{float}
\usepackage{amsmath}
\usepackage{fancyhdr}
\usepackage{hyperref}
\usepackage{mathtools}
            
%%%%  SHORTCUT COMMANDS  %%%%
\newcommand{\ds}{\displaystyle}
\newcommand{\Z}{\mathbb{Z}}
\newcommand{\T}{\mathcal{T}}
\newcommand{\arc}{\rightarrow}
\newcommand{\R}{\mathbb{R}}
\newcommand{\RP}{\mathbb{R}(+)}
\newcommand{\Rs}{\mathbb{R}^{**}}
\newcommand{\C}{\mathbb{C}}
\newcommand{\E}{\mathbb{E}}
\newcommand{\B}{\mathcal{B}}
\newcommand{\PX}{\mathcal{P}(X)}
\newcommand*\rot{\rotatebox{90}}
\newcommand*\OK{\ding{51}}
\newcommand{\N}{\mathbb{N}}
\newcommand{\Q}{\mathbb{Q}}
\newcommand{\Answer}{\vspace{2mm}\textbf{\underline{Answer}}\\}
\newcolumntype{L}{>{$}l<{$}}
\newcolumntype{C}{>{$}c<{$}}
\newcommand{\GL}{\text{GL}}
\newcommand{\SL}{\text{SL}}

\newcommand{\stirling}[2]{\genfrac{\{}{\}}{0pt}{}{#1}{#2}}

\pagestyle{fancy}
\fancyhf{}
\renewcommand{\sectionmark}[1]{\markright{\thesection.\ #1}}
\lhead{\fancyplain{}{}} 
\fancyhead[RE,RO]{Math 4451, Spring 2026}
\fancyfoot[RE,RO]{\thepage}
\IfFileExists{upquote.sty}{\usepackage{upquote}}{}
\begin{document}
\begin{flushright}
\begin{minipage}{.33\textwidth}
\rightline{YOUR NAME}
\rightline{\href{mailto:YOUR EMAIL}{YOUR EMAIL}}
\rightline{\today}
\end{minipage}
\end{flushright}

\begin{center}
{\large{\textbf{Homework 1}}}
\end{center}

\begin{enumerate}
    \item (Two-Envelope paradox) A teacher puts an unknown amount of money in one envelope, and twice that amount in another. He asks you to pick one envelope at random, open it, and then decide if you want to exchange it with the other. You pick one (randomly), open it, and see the outcome $X = x$ dollars. You reason: ``Suppose that $Y$ is the content of the other envelop, then $Y$ is either $x / 2$, or $2x$, each with probability $0.5$; the expected amount of money in $Y$ is: $E[Y] = 0.5(2x) + 0.5(x/2) = 5x/4$, which is bigger than your current value $x$". Should you switch the envelope? Consider the following information before you decide:
    \begin{itemize}
      \item This reasoning holds for all values of $x$. This means that you actually \emph{do not even need to open the first envelope}, and yet this reasoning implies you would still want to switch! Once you get the second envelope, the same logic applies, so you should switch it back (perhaps doing this infinitely).
    \end{itemize}
    \textbf{(2 points)} Come up with either a Bayesian or Frequentest solution to this paradox. You can just answer this ``philosophically", i.e., no need for specific computations. There's not necessarily a right or wrong answer here.
    
    \item (4 points) Suppose that $X$ is a discrete random variable with:
    $$
    P(X = x) = \begin{cases} \frac{2}{3}\theta & x = 0 \\ \frac{1}{3}\theta & x = 1 \\ \frac{2}{3}(1 - \theta) & x = 2 \\ \frac{1}{3}(1 - \theta) & x = 3 \\\end{cases}
    $$
    where $0 \leq \theta \leq 1$. Suppose we observe 10 independent observations from such a distribution: $(3, 0, 2, 1, 3, 2, 1, 0, 2, 1)$.
    (We'll return to this same distribution later)
    \begin{itemize}
      \item (1 point) Verify that this is indeed a valid pmf. 
      \item (2 points) Find the method of moments estimate of $\theta$.
      \item (1 point) Find an approximate standard error for your estimate.
    \end{itemize}
    
    \item (4 points) In a study of bird-feeding behavior, the scientist measured the number of hops between flights for several birds. The results are given in Table~\ref{tab:hops}.
    
    \begin{table}[ht]
    \centering
    \begin{tabular}{cr}\hline
      Number of hops & Count \\\hline
      1 & 48\\
      2 & 31\\
      3 & 20\\
      4 & 9\\
      5 & 6\\
      6 & 5\\
      7 & 4\\
      8 & 2\\
      9 & 1\\
      10 & 1\\
      11 & 2\\
      12 & 1\\\hline
    \end{tabular}
    \caption{\label{tab:hops}Number of hops between flights.}
    \end{table}
    % 
    \begin{itemize}
      \item (2 points) Consider modeling the number of hops between two-flights $(X_i)$ as a Geometric random variable. Derive the method of moment estimator of the relevant model parameters.
      \item (1 point) Do you think this is a good model for the given data? Why or why not?
    \end{itemize}


    \item (4 points) The National Bureau of Standards collected data about alpha-particle emissions for americium 241, a type of synthetic radioactive metal useful for making smoke detectors. We wish to model the number of particles emitted per second $X_i$ using a Poisson$(\lambda)$ distribution.
    
    Real-data collected by the agency are provided in terms of counting the number of particles $(n)$ emitted in 1-second intervals, observed over an approximate 203 minute observation period (total of 12,169 1-second intervals). The data are given in Table~\ref{tab:alpha}.
    
    \begin{table}[ht]
    \centering
    \begin{tabular}{lr}\hline
      $n$ & Number of Intervals \\\hline
      0 & 5267 \\
      1 & 4436 \\
      2 & 1800 \\
      3 & 534 \\
      4 & 111 \\
      5+ & 21 \\\hline
    \end{tabular}
    \caption{\label{tab:alpha}Number of 1-second intervals that observed $n$ alpha particle emissions.}
    \end{table}
    
    \begin{itemize}
      \item (1 point) Using the method-of-moments estimation technique, estimate the value of $\lambda$, the rate of particles emitted per second. 
      \item (2 points) Using your fitted model, what are the expected counts of intervals for each $n$? How does this compare to what was observed?
      \item (1 point, Open ended) Do you think this is an appropriate model for this data? Why or why not? Consider answering this in terms of a simulation study: simulate several times from your fitted model, and compare the results to the observed data.
    \end{itemize}
\end{enumerate}


\end{document}
